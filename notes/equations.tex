% !TEX encoding = UTF-8 Unicode
% !TEX TS-program = xelatex

% default font size and paper size
\documentclass[10pt, a4paper]{article}

% set the page margins
\usepackage[a4paper, margin=0.7in]{geometry}

% setup the language
\usepackage[english]{babel}

% define hyphenation rules
\hyphenation{some-long-word}

\usepackage{style}

\usepackage[backref=true]{biblatex}
\addbibresource{references.bib}

\begin{document}

\todaydate{}
\topic{Designing Kalman Filters}

\section*{Overview}
The Kalman Filter involves two steps: \textit{predict} and \textit{correct}. Each of these steps is detailed below.

\subsection*{\textit{Predict}}
This step uses the process model to predict the current state based on the previous state and the control input at the current time.
The equations describing this idea are
\begin{align}
  \bar{\mathbf{x}} &= \mathbf{Ax} + \mathbf{Bu} \\
  \bar{\mathbf{P}} &= \mathbf{APA}^\top
\end{align}
where $\mathbf{x}$ is the state vector, $\mathbf{A}$ is the state transition matrix, $\mathbf{B}$ is the control matrix, $\mathbf{u}$ is the control input and $\mathbf{P}$ is the state covariance matrix.

\subsection*{\textit{Correct}}
This step corrects the current prediction using the current measurement as follows.
\begin{align}
  \mathbf{x} &= \bar{\mathbf{x}} + \mathbf{Ky} \\
  \mathbf{P} &= (\mathbf{I} - \mathbf{KH})\bar{\mathbf{P}}
\end{align}

where

\begin{align*}
  \mathbf{y} &= \mathbf{z} - \mathbf{H}\bar{\mathbf{x}} \\
  \mathbf{K} &= \bar{\mathbf{P}}\mathbf{H}^\top (\mathbf{H}\bar{\mathbf{P}}\mathbf{H}^\top + \mathbf{R})^{-1}
\end{align*}

and $\mathbf{H}$ is the measurement matrix, $\mathbf{R}$ is the measurement noise covariance and $\mathbf{I}$ is the identity matrix.
The above equations assume that a multivariate Gaussian is used to represent the process model, measurement model and the state.


\section{Constant Velocity Model}
The state transition model of an object under the constant velocity model is given by
\[
\begin{bmatrix}
x \\
\dot{x} \\
y \\
\dot{y}
\end{bmatrix}
= 
\begin{bmatrix}
1 & \Delta t & 0 & 0 \\
0 & 1 & 0 & 0 \\
0 & 0 & 1 & \Delta t \\
0 & 0 & 0 & 1
\end{bmatrix}
\begin{bmatrix}
x \\
\dot{x} \\
y \\
\dot{y}
\end{bmatrix}
\]

The measurement only involves position, however incorporating velocity in the state model improves the estimates.

\section{Constant Acceleration Model}
A position of a ball thrown in vacuum id given by

\begin{align}
    y &= \frac{g}{2} t^2 + v_{y_0}t + y_0 \\
    x &= v_{x_0}t + x_0
\end{align}

where $g$ is the gravitational constant, $t$ is time, $v_{x0}$ and $v_{y0}$ are the initial velocities in the x and y plane.
The state transition model can be similar to the constant velocity model and the gravitational acceleration can be incorporated in the control input $\mathbf{u}$.

\printbibliography

\end{document}
